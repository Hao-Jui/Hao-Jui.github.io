\documentclass[10pt,floatfix,a4paper]{article}
\usepackage{mathrsfs,amsfonts,amssymb,amsmath}
\usepackage[normalem]{ulem}
\usepackage{graphicx}
\usepackage[linktocpage,breaklinks]{hyperref}
\usepackage[capitalize]{cleveref}
\usepackage[usenames,dvipsnames]{xcolor}
\hypersetup{colorlinks=true,
            citecolor=NavyBlue,
            linkcolor=magenta,
            urlcolor=magenta}
\usepackage{multirow,array}
\DeclareMathAlphabet{\pazocal}{OMS}{zplm}{m}{n}

\usepackage{wrapfig}
\usepackage[capitalize]{cleveref}
\usepackage[legalpaper, margin=1in]{geometry}
%\usepackage[symbol]{footmisc} % activate the use of * for electronic address

\usepackage[sorting=none,style=numeric-comp,natbib]{biblatex}
\addbibresource{references.bib}

\usepackage{eurosym}
%\linespread{1}
%\pagestyle{plain}

\begin{document}
\begin{center}
  {\LARGE \bf CURRICULUM VITAE} \\[2ex]
  {\Large \bf Hao-Jui Kuan}
\end{center}

%\vspace{5mm}

\section*{Personal information}
\begin{tabbing}
  \hspace*{5mm} \= \hspace*{2.3cm} \= \hspace*{10cm} \\[-3ex]
  \> Date of birth\hspace{1.2mm}:  \> 13 July 1995\\
  \> Nationality\hspace{4mm}: \> Taiwan\\
  \> Email/web\hspace{4.3mm}: \> hao-jui.kuan@aei.mpg.de, \href{https://hao-jui.github.io}{\textcolor{magenta}{hao-jui.github.io}}\\
  \> Address\hspace{8.7mm}: \> Albert-Einstein-Institut, Am M{\"u}hlenberg 1, 14476 Potsdam
\end{tabbing}


\section*{Professional experience}
\begin{tabbing}
  \hspace*{5mm} \= \hspace*{2.3cm} \= \hspace*{10cm} \\[-3ex]
  \> 2023 - \> Postdoc in the Computational Relativistic Astrophysics division at the Max Planck Institute \\
  \> \> for Gravitational Physics (Albert Einstein Institute)
\end{tabbing}

\section*{Education}
\begin{tabbing}
  \hspace*{5mm} \= \hspace*{2.3cm} \= \hspace*{10cm} \\[-3ex]
  \> 2019 - 2022 \>  University of T{\"u}bingen, T\"ubingen, Germany. \textit{Advisor: Kostas D.~Kokkotas} \\
  \> \> Ph.D. in Theoretical Astrophysics.  Summa Cum Laude (excellent). \\[1ex]
  \> 2017 - 2022 \> National Tsing Hua University, Taiwan. \textit{Advisor: Chao-Qiang Geng} \\
  \> \> PhD in Physics (joint degree). \\[1ex]
  \> 2013 - 2017 \> National Tsing Hua University, Taiwan. \\
  \> \> B.Sc Double major in Physics and Mathematics.
\end{tabbing}

%\section*{Referee activity}
%\section*{Research interest}
%\begin{itemize}
%    \item Evolutionary study of the pre-merger dynamics of magnetized neutron stars 
%    \item Dynamical aspects of compact objects in alternative theories
%    \item Quasi-normal mode asteroseismology via universal relations, and tidal effects in binaries
%\end{itemize}

\section*{Honors and Awards}
\begin{tabbing}
  \hspace*{5mm} \= \hspace*{2.3cm} \= \hspace*{10cm} \\[-3ex]
  \> 2023\> Honorable Mention in GWIC-Braccini Thesis Prize (\href{https://gwic.ligo.org/thesis-prize.html}{website}) \\
  \> 2023\> Dr. Friedrich F{\"o}rster Prize from University of T{\"u}bingen \\
  \> 2023\> Student Outstanding Paper Award from NCTS (Taiwan)
\end{tabbing}



\section*{Funded Grant Activity}
\begin{tabbing}
  \hspace*{5mm} \= \hspace*{2.3cm} \= \hspace*{10cm} \\[-3ex]
  \> 2020 - 2021 \> ``Neutron stars as gravitational wave sources'', Sandwich-Scholarship Programme,\\
  \> \> DAAD and MOST (Taiwan), \euro{13,500}. \\
  \> 2017 - 2020 \> Presidential Scholarship of National Tsing Hua University (Taiwan), converted as $\sim$18,700 USD.
\end{tabbing}

\iffalse
\section*{Scientific summary}
\begin{tabular}{ l l r l}
%\hline
%\hline
  Plenary conference talk 			& ... & 1  \\  
  Invited seminar         			& ... & 9  \\
  Conference talk/poster 	& ... & 10  \\
  Number of first author articles 	& ... & 11 \\
  Refereed Articles 				& ... & 17 \\
  Preprints/journal sumissoions 		& ... & 3  \\
  h index (HEP-SPIRES) 				& ... & 10 \\
\end{tabular}
\fi

\section*{Reviewing Activities}
\begin{tabbing}
  \hspace*{5mm} \= \hspace*{2.3cm} \= \hspace*{10cm} \\[-3ex]
  \> {\bf Journals} \> Physical Review D, International Journal of Modern Physics D, European Physical Journal C, \\
  \> \> Monthly Notices of the Royal Astronomical Society
\end{tabbing}




\section*{List of Presentations}
\hspace*{4mm}
{\large \bf \textcolor{Violet}{Plenary Talks at Conferences and Workshops}}
\begin{tabbing}
  \hspace*{5mm} \= \hspace*{2cm} \= \hspace*{10cm} \\[-3ex]
  %
  \> {\bf 2022} \> \\[1ex]
  \> Sep 10 \> ``\textbf{Premerger Neutron Star Physics}'', Eleventh Aegean summer school, Syros, Greece.
\end{tabbing}



{\large \bf \textcolor{Violet}{Invited Seminars}}
\begin{tabbing}
  \hspace*{5mm} \= \hspace*{2cm} \= \hspace*{10cm} \\[-3ex]
  %
  \> {\bf 2024} \> \\[1ex]
  \> Aug 1 \> ``\textbf{On the finite-size imprints on waveforms of binary neutron star mergers}'', UIUC. \\  
  \> July 11 \> ``\textbf{Imprints of matter and scalar effects on waveforms from binary neutron star mergers}'',\\\
  \> \> online-seminar-in-mathematical-numerical-relativity (\href{https://github.com/Mathematical-Numerical-Relativity/Online-Seminar/wiki}{website}). \\
  \> April 24 \> ``\textbf{Imprints of matter and scalar effects on waveforms from binary neutron star mergers}'',\\
  \> \>  Institute for Gravitational and Subatomic Physics (GRASP), Utrecht University.\\
  \> April 8 \> ``\textbf{(Dynamical) Tidal Effects in the Binary Neutron Star mergers}'',\\
  \> \> AstroParticle and Cosmology laboratory (APC), Paris, France. \\
  %
  \> {\bf 2023} \> \\[1ex]
  \> March 16 \> ``\textbf{Dynamical Scalarization during Neutron Star Mergers in scalar-Gauss-Bonnet Theory}'', \\
  \> \> University of T{\"u}bingen, Germany.\\
  \> February 3 \> ``\textbf{Developing waveform involving dynamical tides}'', \\
  \> \> Academia Sinica, Taipei, Taiwan.\\
  \> February 2 \> ``\textbf{Gravitational Phase Transition in
  Massive Scalar-tensor Theory}'', \\
  \> \> National Center for Theoretical Sciences, Hsinchu, Taiwan.\\
  %
  \> {\bf 2021} \> \\[1ex]
  \> July 27 \>``\textbf{$g$-mode resonances as triggers for precursors of SGRBs}'', \\ 
  \> \> Grav@zon group, Federal University of Pará, Pará, Brazil. \\
  \> Jun 22\>``\textbf{Dynamical formation of scalarized black holes and neutron stars through stellar}\\
  \> \> \textbf{core collapse}'', \\ 
  \> \> cosmo/GW journal club, Johns Hopkins University, Maryland, USA.
\end{tabbing}


{\large \bf \textcolor{Violet}{Contributed Talks}}
\begin{tabbing}
  \hspace*{5mm} \= \hspace*{2cm} \= \hspace*{10cm} \\[-3ex]
  %
  \> {\bf 2024} \> \\[1ex]
  \> Oct 9 \> ``\textbf{On the finite-size imprints on waveforms of binary neutron star mergers}'',\\
  \> \> talk, Observatoire de Paris - site de Meudon, France. \\  
  \> Feb 28 \> ``\textbf{Binary neutron star mergers in massive scalar-tensor theory}'', talk,\\
  \> \> Gravity and Cosmology 2024, Yukawa Institute for Theoretical Physics, Kyoto, Japan.\\
  %
  \> {\bf 2023} \> \\[1ex]
  \> Oct 26 \> ``\textbf{Binary neutron star mergers in massive scalar-tensor theory: an
  adiabatic look}'', poster, \\
  \> \> GravityShapePisa 2023, Pisa, Italy.\\
  \> April 5 \> ``\textbf{Dynamical Scalarization during BNS mergers
  in scalar-Gauss-Bonnet}'', \\
  \> \> CoCoNut meeting, Potsdam, Germany.\\
  \> March 22 \> ``\textbf{Packed Message delivered by Tides in
  Binary Neutron Star Mergers}'', \\
  \> \> SMuK2023, Technical University Dresden, Germany.\\
  %
  \> {\bf 2022} \> \\[1ex]
  \> Sep 08 \> ``\textbf{Gravitaional Phase Transition}'', \\
  \> \> Eleventh Aegean summer school, Syros, Greece.\\
  \> May 16-19 \> ``\textbf{Tidal effects in the pre-merger stage of coalescing binary neutron stars}'', e-poster, \\
  \> \> PHAROS: The multi-messenger physics and astrophysics of neutron stars, Roma, Italy.\\
  \> Mar 3 \> ``\textbf{Resonance shattering as triggers for precursors of SGRBs}'', \\
  \> \> DPG Meeting of the Matter and Cosmos section (SMuK), Heidelberg, Germany.\\
  %
  \> {\bf 2021} \> \\[1ex]
  \> Aug30 - Sep3 \> ``\textbf{Tidal $g$-mode resonances in coalescing binaries of neutron stars as triggers for}\\
  \> \> \textbf{precursor flares of short gamma-ray bursts}'', \\ 
  \> \> DPG Meeting of the Matter and Cosmos section (SMuK), Bad Honnef, Germany. \\
  \> Feb 24-28\>``\textbf{Inverse-Chirp Imprint of GW in Scalar Tensor Theory}'', \\ 
  \> \> 56th Karpacz Winter School in Theoretical Physics, Karpacz, Poland.
\end{tabbing}


%%%%%%%%%%%%%%%%%%%%%%%%%%%%%%
\section*{Teaching Activities}
\qquad {\Large \bf Teaching assistant}
\begin{tabbing}
  \hspace*{10mm} \= \hspace*{2.3cm} \= \hspace*{10cm} \\[-3ex]
  \> \textbf{At NTHU} \> General Relativity (2016 Fall, 2018 Fall)\\
  \> \> Mathematical Physics (2017-2018) \\
  \> \> Thermal Physics (2016-2017) \\
  \> \> Calculus (2018-2019) \\
  \> \textbf{At IAAT} \> Physical Practical (2021 Summer \& Winter)\\
  \> \> Grundkurs Electromagnetism (2022 Summer)
\end{tabbing}
%%%%%%%%%%%%%%%%%%%%%%%%%%%%%%


%\section*{Miscellaneous}
%\begin{tabbing}
%\hspace*{3.cm} \= \hspace*{10cm} \\[-3ex]
%\textbf{Computer skills} \> Programming: Python, Fortran\\ 
%\>Scientific software: Matlab, Mathematica, Gnuplot\\ 
%\textbf{Language} \> English (fluent), Mandarin (Mother tongue), Japanese (fluent)
%\end{tabbing}

\iffalse
\section*{References}
\begin{itemize}
    \item \textbf{Prof. Dr. Kostas Kokkotas}\\
    Theoretical Astrophysics (IAAT)\\ 
    Eberhard Karls Universit\"at T\"ubingen \\
    Auf der Morgenstelle 10, 72076 T\"ubingen, Germany\\
    e-mail: kostas.kokkotas@uni-tuebingen.de
    \vspace{2mm}
    \item \textbf{Prof. Dr. Masaru Shibata}\\
    Max-Planck-Institut f{\"u}r Gravitationsphysik 
	(Albert-Einstein-Institut) \\
    Am M{\"u}hlenberg 1, 14476 Potsdam, Germany\\
    e-mail: masaru.shibata@aei.mpg.de
    \vspace{2mm}
    \item \textbf{Dr. Kenta Kiuchi}\\
    Max-Planck-Institut f{\"u}r Gravitationsphysik 
	(Albert-Einstein-Institut) \\
    Am M{\"u}hlenberg 1, 14476 Potsdam, Germany\\
    e-mail: kenta.kiuchi@aei.mpg.de
    \vspace{2mm}
    \item \textbf{Dr. Arthur Suvorov}\\
    Manly Astrophysics\\
    5/41-42 East Esplanade, Manly, NSW 2095, Australia\\
    e-mail: arthur.suvorov@manlyastrophysics.org
    \vspace{2mm}
    \item \textbf{Dr. Daniela Doneva}\\
    Theoretical Astrophysics (IAAT)\\ 
    Eberhard Karls Universit\"at T\"ubingen \\
    Auf der Morgenstelle 10, 72076 T\"ubingen, Germany\\
    e-mail: daniela.doneva@uni-tuebingen.de
\end{itemize}

\vspace{2cm}
\Large
\begingroup
  \small
  \renewcommand*{\arraystretch}{1.5}
  \noindent
  \begin{tabular}[t]{@{}l@{}}
  
  \end{tabular}\hfill
  \begin{tabular}[t]{c}
    \textbf{Golm}\\
    \textbf{\today}\\
    \\
    \textbf{Hao-Jui Kuan}\\
  \end{tabular}
\endgroup
\fi

\end{document}


\end{document}


