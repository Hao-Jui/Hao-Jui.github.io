\documentclass[11pt,a4paper,sans]{moderncv}

\moderncvstyle{classic}                            
%'casual', 'classic', 'oldstyle' and 'banking'
\moderncvcolor{blue}
%\renewcommand{\familydefault}{\sfdefault}

\nopagenumbers{}

% character encoding
\usepackage[utf8]{inputenc}
\usepackage{fontawesome}
\usepackage{tabularx}
\usepackage{ragged2e}
%\usepackage{CJKutf8}


\usepackage[colorlinks=true]{hyperref}
\hypersetup{citecolor=cyan,linkcolor=red}

\usepackage[scale=0.8]{geometry}
\usepackage{multicol}
%\setlength{\hintscolumnwidth}{3cm}               
% if you want to change the width of the column with the dates
%\setlength{\makecvtitlenamewidth}{10cm}           
% for the 'classic' style, if you want to force the width allocated to your name and avoid line breaks. be careful though, the length is normally calculated to avoid any overlap with your personal info; use this at your own typographical risks...

\usepackage{import}

% personal data
\name{Hao-Jui}{Kuan}
\title{Curriculum Vitae}                 
  
\setlength{\tabcolsep}{12pt}

%----------------------------------------------------------------------------------
%            content
%----------------------------------------------------------------------------------
\begin{document}
%\begin{CJK*}{UTF8}{gbsn}
%-----       resume       ---------------------------------------------------------
\makecvtitle
\vspace{-1cm}
\section{Personal information}
\begin{tabbing}
\hspace*{2.3cm} \= \hspace*{10cm} \\[-3ex]
Date of birth \> 13 July 1995\\
Nationality \> Taiwan\\
Email/web \> hao-jui.kuan@aei.mpg.de, \href{https://hao-jui.github.io}{\textcolor{magenta}{hao-jui.github.io}}\\
Address\> Albert-Einstein-Institut, Am M{\"u}hlenberg 1, 14476 Potsdam
\end{tabbing}

\section{Education}
\begin{tabbing}
\hspace*{2.3cm} \= \hspace*{10cm} \\[-3ex]
2019 - 2022 \> \textbf{PhD in Theoretical Astrophysics with  ``summa cum laude'' (excellent)}\\
\> University of T{\"u}bingen, T\"ubingen, Germany. \textit{Advisor: Kostas D.~Kokkotas} \\
%\> Main arguments: Quasi-normal modes (QNMs) in Magnetars, evolutionary study of binaries, \\
%\> and QNMs asteroseismology. \\
%
2017 - 2022 \> \textbf{PhD in Physics (joint degree)}\\
\> National Tsing Hua University, Hsinchu, Taiwan. \textit{Advisor: Chao-Qiang Geng} \\
%\> Main arguments: Hydrodynamics simulation in alternative gravities \\
%
2013 - 2017 \> \textbf{B.Sc Double major in Physics and Mathematics}
\end{tabbing}


\section{Professional experience}
\begin{tabbing}
\hspace*{2.3cm} \= \hspace*{10cm} \\[-3ex]
2023 - \> Postdoc in the Computational Relativistic Astrophysics division at the Max Planck Institute \\
\> for Gravitational Physics (Albert Einstein Institute)
\end{tabbing}

%\section{Referee activity}
%\section{Research interest}
%\begin{itemize}
%    \item Evolutionary study of the pre-merger dynamics of magnetized neutron stars 
%    \item Dynamical aspects of compact objects in alternative theories
%    \item Quasi-normal mode asteroseismology via universal relations, and tidal effects in binaries
%\end{itemize}

\section{Awards, Honors, Fellowships \& Grants}
%%%%%%%%%%%%%%%%%%%%%%%%%%%%%%
\textcolor{cyan}{Grants}
\begin{tabbing}
\hspace*{2.3cm} \= \hspace*{10cm} \\[-3ex]
2020 - 2021 \> Principal Investigator of Sandwich-Scholarship Programme;\\
\> 13,500 euro, funded by Deutscher Akademischeer Austauschdienst (DAAD), and Ministry of \\
\> Science and Technology, Taiwan (MOST) with funding ID being JYP 109-2927-I-007-503.
\end{tabbing}

%%%%%%%%%%%%%%%%%%%%%%%%%%%%%%
\textcolor{cyan}{Awards/Honors}
\begin{tabbing}
\hspace*{2.3cm} \= \hspace*{10cm} \\[-3ex]
2023\> Honorable Mention in GWIC-Braccini Thesis Prize (\href{https://gwic.ligo.org/thesis-prize.html}{website}) \\
2023\> Dr. Friedrich F{\"o}rster Prize from University of T{\"u}bingen\\
2023\> Student Outstanding Paper Award from NCTS, Taiwan
\end{tabbing}

%%%%%%%%%%%%%%%%%%%%%%%%%%%%%%
\textcolor{cyan}{Scholarship}
\begin{tabbing}
\hspace*{2.3cm} \= \hspace*{10cm} \\[-3ex]
2017 - 2020 \> Presidential Scholarship of National Tsing Hua University, Taiwan
\end{tabbing}



\section{Scientific summary}
\begin{tabular}{ l l r l}
%\hline
%\hline
  Plenary conference talk 			& ... & 1  \\  
  Invited seminar         			& ... & 9  \\
  Conference talk/poster 	& ... & 10  \\
  Number of first author articles 	& ... & 11 \\
  Refereed Articles 				& ... & 17 \\
  Preprints/journal sumissoions 		& ... & 3  \\
  h index (HEP-SPIRES) 				& ... & 10 \\
\end{tabular}

\section{Reviewing Activities}
Reviewer for Journals: Physical Review D, International Journal of Modern Physics D, Monthly Notices of the Royal Astronomical Society, European Physical Journal C




\section{Full Presentations/Seminar List}

%%%%%%%%%%%%%%%%%%%%%%%%%%%%%%
{\Large \textcolor{cyan}{2024}}

\begin{tabbing}
\hspace*{2.3cm} \= \hspace*{10cm} \\[-3ex]
Oct 9 \> ``\textbf{On the finite-size imprints on waveforms of binary neutron star mergers}'',\\
\> talk, Observatoire de Paris - site de Meudon, France. \\
Aug 1 \> ``\textbf{On the finite-size imprints on waveforms of binary neutron star mergers}'',\\
\> invited seminar, UIUC. \\
July 11 \> ``\textbf{Imprints of matter and scalar effects on waveforms from binary neutron star mergers}'',\\
\>  invited seminar, online-seminar-in-mathematical-numerical-relativity (\href{https://github.com/Mathematical-Numerical-Relativity/Online-Seminar/wiki}{website}). \\
April 24 \> ``\textbf{Imprints of matter and scalar effects on waveforms from binary neutron star mergers}'',\\
\>  invited seminar, Institute for Gravitational and Subatomic Physics (GRASP), \\
\> Department of Physics, Utrecht University\\
April 8 \> ``\textbf{(Dynamical) Tidal Effects in the Binary Neutron Star mergers}'',\\
\> invited seminar, the gravitation group in the AstroParticle and Cosmology laboratory (APC),\\
\> Paris, France (remotely).\\
Feb 28 \> ``\textbf{Binary neutron star mergers in massive scalar-tensor theory}'', talk,\\
\> Gravity and Cosmology 2024, Yukawa Institute for Theoretical Physics, Kyoto, Japan.
\end{tabbing}
%%%%%%%%%%%%%%%%%%%%%%%%%%%%%%
{\Large \textcolor{cyan}{2023}}
\begin{tabbing}
\hspace*{2.3cm} \= \hspace*{10cm} \\[-3ex]
Oct 26 \> ``\textbf{Binary neutron star mergers in massive scalar-tensor theory: an
adiabatic look}'', poster, \\
\> GravityShapePisa 2023, Pisa, Italy.\\
%
April 5 \> ``\textbf{Dynamical Scalarization during BNS mergers
in scalar-Gauss-Bonnet}'', talk, \\
\> CoCoNut meeting, Potsdam, Germany.\\
%
March 22 \> ``\textbf{Packed Message delivered by Tides in
Binary Neutron Star Mergers}'', parallel talk, \\
\> SMuK2023, Technical University Dresden, Germany.\\
%
March 16 \> ``\textbf{Dynamical Scalarization during Neutron Star Mergers in scalar-Gauss-Bonnet Theory}'', \\
\> invited seminar, University of T{\"u}bingen, Germany.\\
%
February 3 \> ``\textbf{Developing waveform involving dynamical tides}'', invited seminar, \\
\> Academia Sinica, Taipei, Taiwan.\\
%
February 2 \> ``\textbf{Gravitational Phase Transition in
Massive Scalar-tensor Theory}'', invited seminar, \\
\> National Center for Theoretical Sciences, Hsinchu, Taiwan.\\
\end{tabbing}

%%%%%%%%%%%%%%%%%%%%%%%%%%%%%%
{\Large \textcolor{cyan}{2022}}
\begin{tabbing}
\hspace*{2.3cm} \= \hspace*{10cm} \\[-3ex]
Sep 10 \> ``\textbf{Premerger Neutron Star Physics}'', plenary talk, \\
\> Eleventh Aegean summer school, Syros, Greece.\\
%
Sep 08 \> ``\textbf{Gravitaional Phase Transition}'', parallel talk, \\
\> Eleventh Aegean summer school, Syros, Greece.\\
%
May 16-19 \> ``\textbf{Tidal effects in the pre-merger stage of coalescing binary neutron stars}'', e-poster, \\
\> PHAROS: The multi-messenger physics and astrophysics of neutron stars, Roma, Italy.\\
%
Mar 3 \> ``\textbf{Resonance shattering as triggers for precursors of SGRBs}'', parallel talk, \\
\> DPG Meeting of the Matter and Cosmos Section (SMuK), Heidelberg, Germany. (remotely)
\end{tabbing}

%%%%%%%%%%%%%%%%%%%%%%%%%%%%%%
{\Large \textcolor{cyan}{2021}}
\begin{tabbing}
\hspace*{2.3cm} \= \hspace*{10cm} \\[-3ex]
Aug30 - Sep3 \> ``\textbf{Tidal $g$-mode resonances in coalescing binaries of neutron stars as triggers for}\\
\> \textbf{precursor flares of short gamma-ray bursts}'', parallel talk\\ 
\> DPG Meeting of the Matter and Cosmos Section (SMuK), Bad Honnef, Germany. (remotely)\\
%
July 27 \>``\textbf{$g$-mode resonances as triggers for precursors of SGRBs}'', invited talk\\ 
\> Grav@zon group seminar, Federal University of Pará, Pará, Brazil. (remotely) \\
%
Jun 22\>``\textbf{Dynamical formation of scalarized black holes and neutron stars through stellar}\\
\> \textbf{core collapse}'', invited seminar\\ 
\> cosmo/GW journal club, Johns Hopkins University, Maryland, USA. (remotely)
\end{tabbing}

%%%%%%%%%%%%%%%%%%%%%%%%%%%%%%
{\Large \textcolor{cyan}{2020}}
\begin{tabbing}
\hspace*{2.3cm} \= \hspace*{10cm} \\[-3ex]
Feb 24-28\>``\textbf{Inverse-Chirp Imprint of GW in Scalar Tensor Theory}'', parallel talk\\ 
\>56th Karpacz Winter School in Theoretical Physics, Karpacz, Poland.
\end{tabbing}

\section{Teaching experience}
\begin{tabbing}
\hspace*{2.3cm} \= \hspace*{10cm} \\[-3ex]
\textbf{At NTHU} \> Teaching assistant for General Relativity (2016 Fall, 2018 Fall), Mathematical Physics (2017-2018),\\
\> Thermal Physics (2016-2017), Calculus (2018-2019) \\
\textbf{At IAAT} \> Teaching assistant for Physical Practical (2021 Summer \& Winter)\\
\> Teaching assistant for Grundkurs Electromagnetism (2022 Summer)
\end{tabbing}
%%%%%%%%%%%%%%%%%%%%%%%%%%%%%%


%\section{Miscellaneous}
%\begin{tabbing}
%\hspace*{3.cm} \= \hspace*{10cm} \\[-3ex]
%\textbf{Computer skills} \> Programming: Python, Fortran\\ 
%\>Scientific software: Matlab, Mathematica, Gnuplot\\ 
%\textbf{Language} \> English (fluent), Mandarin (Mother tongue), Japanese (fluent)
%\end{tabbing}

\section{Publication List}
\noindent
{\Large \textsc{Refereed Article} }
\\
\begin{enumerate}
	%
	\item V. Brdar, T. Cheng, \textbf{H.-J. Kuan}, and Y.-Y. Li. Magnetar-powered neutrinos and magnetic moment signatures at IceCube. \href{https://iopscience.iop.org/article/10.1088/1475-7516/2024/07/026}{JCAP 07:026, July 2024}. 
	%
	\item A.~G.~Suvorov, \textbf{H.-J.~Kuan}, Alexis Reboul-Salze and K.~D.~Kokkotas. Magnetic amplification in pre-merger neutron stars through resonance-induced magnetorotational instabilities.  \href{https://journals.aps.org/prd/abstract/10.1103/PhysRevD.109.103023}{Phys.Rev.D 109:103023, May 2024}.
	%
	\item \textbf{H.-J.~Kuan} and K.~D.~Kokkotas. Last three seconds: Packed message delivered by tides in binary neutron star mergers. \href{https://journals.aps.org/prd/abstract/10.1103/PhysRevD.108.063026}{Phys. Rev. D 108:063026, September 2023}. 
	%
	\item \textbf{H.-J.~Kuan}, K.~V.~Van Aelst, A.~T.~L.~Lam and M.~Shibata. Binary neutron star mergers in massive scalar-tensor theory: Quasiequilibrium states and dynamical enhancement of the scalarization. \href{https://journals.aps.org/prd/abstract/10.1103/PhysRevD.108.064057}{Phys. Rev. D 108:064057, September 2023}.
	%
	\item \textbf{H.-J.~Kuan}, A.~G.~Suvorov and K.~D.~Kokkotas. Measuring spin in coalescing binaries of neutron stars showing double precursors. \href{	https://doi.org/10.1051/0004-6361/202346658}{Astron. Astrophys., 676(2):A59, June 2023}.
	%
	\item \textbf{H.-J.~Kuan}, A.~T.~L.~Lam, D.~D.~Doneva, S.~S.~Yazadjiev, M.~Shibata and K.~Kiuchi. Dynamical scalarization during neutron star mergers in scalar-Gauss-Bonnet theory. \href{https://journals.aps.org/prd/abstract/10.1103/PhysRevD.108.063033}{Phys. Rev. D 108:063033, September 2023}.
	%
	\item \textbf{H.-J.~Kuan} and K.~D.~Kokkotas. $f$-mode imprints on gravitational waves from coalescing binaries involving aligned spinning neutron stars. \href{https://journals.aps.org/prd/abstract/10.1103/PhysRevD.106.064052}{Phys. Rev. D 106:064052, September 2022}.
	%
	\item \textbf{H.-J.~Kuan}, A.~G.~Suvorov, D.~D.~Doneva and S.~S.~Yazadjiev. Gravitational Waves from Accretion-Induced Descalarization in Massive Scalar-Tensor Theory. \href{https://journals.aps.org/prl/abstract/10.1103/PhysRevLett.129.121104}{Phys. Rev. Lett. 129:121104, September 2022}.
	%
	\item A.~G.~Suvorov, \textbf{H.-J.~Kuan} and K.~D.~Kokkotas. Quasi-periodic oscillations in precursor flares via seismic aftershocks from resonant shattering. \href{https://www.aanda.org/articles/aa/full_html/2022/08/aa44082-22/aa44082-22.html}{Astron. Astrophys. 664:A177, August 2022}.
	%
	\item \textbf{H.-J.~Kuan}, C.~J.~Kr{\"u}ger, A.~G.~Suvorov and K.~D.~Kokkotas. Constraining equation of state groups from $g$-mode asteroseismology. \href{https://doi.org/10.1093/mnras/stac1101}{MNRAS, 513(3):4045-4056, April 2022}.
	%
	\item \textbf{H.-J.~Kuan}, J.~Singh, D.~D.~Doneva, S.~S.~Yazadjiev, and K.~D.~Kokkotas. Nonlinear evolution and nonuniqueness of scalarized neutron stars. Phys. Rev. D, 104:124013, December 2021. \href{https://doi.org/10.1103/PhysRevD.104.124013}{10.1103/PhysRevD.104.124013}.
	%
	\item \textbf{H.-J.~Kuan}, A.~G.~Suvorov and K.~D.~Kokkotas. General-relativistic treatment of tidal g-mode resonances in coalescing binaries of neutron stars. II. As triggers for precursor flares of short gamma-ray bursts. \href{https://doi.org/10.1093/mnras/stab2658}{MNRAS, 508(2):1732-1744, December 2021}.
	%
	\item D.~Huang, C.~Q.~Geng, and \textbf{H.-J.~Kuan}. Scalar gravitational wave signals from core collapse in massive scalar-tensor gravity with triple-scalar interactions. \href{https://doi.org/10.1088/1361-6382/ac35ab}{Class. Quant. Grav., 38:245006, November 2021}.
	%
	\item \textbf{H.-J.~Kuan}, D.~D.~Doneva, and S.~S.~Yazadjiev. Dynamical Formation of Scalarized Black Holes and Neutron Stars through Stellar Core Collapse. \href{https://doi.org/10.1103/PhysRevLett.127.161103}{Phys. Rev. Lett., 127:161103, October 2021}.
	%
	\item \textbf{H.-J.~Kuan}, A.~G.~Suvorov, and K.~D.~Kokkotas. General-relativistic treatment of tidal g-mode resonances in coalescing binaries of neutron stars - I. Theoretical framework and crust breaking. \href{https://doi.org/10.1093/mnras/stab1898}{MNRAS, 506(2):2985–2998, September 2021}.
	%
	\item C.~Q.~Geng, \textbf{H.-J.~Kuan}, and L.~W.~Luo. Inverse-chirp imprint of gravitational wave signals in scalar tensor theory. \href{https://doi.org/10.1140/epjc/s10052-020-8359-y}{Eur. Phys. J. C, 80:780, August 2020}.
	%
	\item C.~Q.~Geng, \textbf{H.-J.~Kuan}, and L.~W.~Luo. Viable Constraint on Scalar Field in Scalar-Tensor Theory.	\href{https://doi.org/10.1088/1361-6382/ab86fb}{Class. Quant. Grav., 37:115001, May 2020}.
\end{enumerate}

\vspace{1cm}
\noindent
{\Large \textsc{Pre-print} }
\\
\begin{enumerate}
	\item A. T.-L. Lam, Yong Gao, \textbf{H.-J. Kuan}, M. Shibata, K. Van Aelst, K. Kiuchi. Accessing universal relations of binary neutron star waveforms in massive scalar-tensor theory. \href{https://arxiv.org/abs/2410.00137 }{arXiv:2410.00137 }
	\item A. G. Suvorov, \textbf{H.-J. Kuan}, K. D. Kokkotas. Premerger phenomena in neutron-star binary coalescences. \href{https://arxiv.org/abs/2408.16283}{arXiv:2408.16283 }
	\item A. T.-L. Lam, \textbf{H.-J. Kuan}, M. Shibata, K. Van Aelst, K. Kiuchi. Binary neutron star mergers in massive scalar-tensor theory: Properties of post-merger remnants. \href{https://arxiv.org/abs/2406.05211}{arXiv:2406.05211}
\end{enumerate}




\section{References}
\begin{itemize}
    \item \textbf{Prof. Dr. Kostas Kokkotas}\\
    Theoretical Astrophysics (IAAT)\\ 
    Eberhard Karls Universit\"at T\"ubingen \\
    Auf der Morgenstelle 10, 72076 T\"ubingen, Germany\\
    e-mail: kostas.kokkotas@uni-tuebingen.de
    \vspace{2mm}
    \item \textbf{Prof. Dr. Masaru Shibata}\\
    Max-Planck-Institut f{\"u}r Gravitationsphysik 
	(Albert-Einstein-Institut) \\
    Am M{\"u}hlenberg 1, 14476 Potsdam, Germany\\
    e-mail: masaru.shibata@aei.mpg.de
    \vspace{2mm}
    \item \textbf{Dr. Kenta Kiuchi}\\
    Max-Planck-Institut f{\"u}r Gravitationsphysik 
	(Albert-Einstein-Institut) \\
    Am M{\"u}hlenberg 1, 14476 Potsdam, Germany\\
    e-mail: kenta.kiuchi@aei.mpg.de
    \vspace{2mm}
    \item \textbf{Dr. Arthur Suvorov}\\
    Manly Astrophysics\\
    5/41-42 East Esplanade, Manly, NSW 2095, Australia\\
    e-mail: arthur.suvorov@manlyastrophysics.org
    \vspace{2mm}
    \item \textbf{Dr. Daniela Doneva}\\
    Theoretical Astrophysics (IAAT)\\ 
    Eberhard Karls Universit\"at T\"ubingen \\
    Auf der Morgenstelle 10, 72076 T\"ubingen, Germany\\
    e-mail: daniela.doneva@uni-tuebingen.de
\end{itemize}

\vspace{2cm}
\Large
\begingroup
  \small
  \renewcommand*{\arraystretch}{1.5}
  \noindent
  \begin{tabular}[t]{@{}l@{}}
  
  \end{tabular}\hfill
  \begin{tabular}[t]{c}
    \textbf{Golm}\\
    \textbf{\today}\\
    \\
    \textbf{Hao-Jui Kuan}\\
  \end{tabular}
\endgroup
\end{document}


\end{document}


